% LaTeX file for resume 
% This file uses the resume document class (res.cls)

\documentclass[margin]{res} 
% the margin option causes section titles to appear to the left of body text 
\textwidth=5.2in % increase textwidth to get smaller right margin
%\usepackage{helvetica} % uses helvetica postscript font (download helvetica.sty)
%\usepackage{newcent}   % uses new century schoolbook postscript font 
\usepackage[hidelinks]{hyperref}


\begin{document} 
 
\name{\Large Ian Hunt-Isaak\\[12pt]} % the \\[12pt] adds a blank line after name
\address{ianhuntisaak@g.harvard.edu}

 
\begin{resume} 
	 
	\section{Education} 
	\textbf{Harvard University SEAS:} PhD. Candidate in Applied Physics\\%, Harvard University SEAS \\
	\textbf{Oberlin College:} B.A. with High Honors, May 2017 \\
	\textit{Major}: Physics, \textit{Minor}: Mathematics
	 

	\section{Publications}
	\begin{itemize}
		\item Y. Ijiri, K. L. Krycka, I. Hunt-Isaak, et al.  "Correlated spin canting in ordered core-shell $\mathrm{Fe}_3\mathrm{O}_4/\mathrm{Mn}_x\mathrm{Fe}_{3-x}\mathrm{O}_4$ nanoparticle assemblies." {\it Phys. Rev. B} 99, (2019).
		\item Oberdick, S. D. et al.  "Spin canting across core/shell $\mathrm{Fe}_3\mathrm{O}_4/\mathrm{Mn}_x\mathrm{Fe}_{3-x}\mathrm{O}_4$ nanoparticles." {\it Scientific Reports} 8, 3425 (2018).
		
	\end{itemize}
	
	\section{Research Experience}
		{\bf Harvard University - Hekstra lab}\hfill August 2017 - Present\\ 
	Graduate Student
	
	This is an interdisciplinary lab focused on understanding the dynamics of biological systems and how information flows in biology. Members come from Molecular and Cellular Biology, Chemistry, and Applied Physics.
	\begin{itemize} \itemsep -2pt  % reduce space between items
		\item Developing a method for the measurement of nanosecond electric field pulses inside a protein crystal                                                                                                                                                                                                                                                                                                                                                                                                                                                                                                          
		\item Built and maintained electronics control boxes for the structural biology experiments performed my fellow lab members
		\item Developing Raman Spectroscopy as a probe of cell state to investigate the stress responses of Bacteria and Yeast
	\end{itemize}
	{\bf Oberlin College - Ijiri Lab} \hfill Jan. 2015 - May 2017 \\
	Researcher
	\begin{itemize} \itemsep -2pt  % reduce space between items
		\item Investigated the magnetic structure of Manganese Ferrite Nanoparticles with the cutting edge technique of Polarization Analyzed Small Angle Neutron Scattering
		\item Extended the NIST SANS macros enabling faster data analysis
		\item Wrote Scripts to increase speed of analysis 
		\item Developed python analysis scripts for systematic fitting of hundreds of data files
			\begin{itemize}
				\item \url{github.com/ianhi/OC\_SANS\_MACROS}
			\end{itemize}
		\item Completed an Honors thesis
			\begin{itemize}
				\item Unusual Magnetic Spin Arrangements in Manganese Ferrite Nanoparticle Assemblies\\ \small\url{oberlin.edu/arts-and-sciences/departments/physics-and-astronomy/honors}\normalsize
			\end{itemize} 
	\end{itemize}

	
	{\bf National Institute of Standards and Technology}\hfill Summer  2016\\ 
	Summer Undergraduate Research Fellow
	\begin{itemize} \itemsep -2pt  % reduce space between items
		\item Designed and developed a simulator of X-Ray and Neutron scattering from simulations of proteins using periodic boundary conditions
		\item Reduced computation time of scattering calculation and analysis algorithm on multi-million atom systems 5-6x using NumPy and C++
		\item Improved the SASSIE and SASMOL proejcts code developed and utilized by research for analysis and modeling of biological macromolecules
	\end{itemize}

	{\bf Rutgers University - Relativistic Heavy Ion Group} \hfill Summer  2015\\
	REU Student
	\begin{itemize} \itemsep -2pt  % reduce space between items
		\item Studied the Quark Gluon Plasma through Monte Carlo Simulation
		\item Improved a framework to run Monte Carlo Simulations - \tiny{github.com/ianhi/GeneratorInterface}\normalsize
		\item Investigated the 3/2 Jet Ratio in Lead Ion Collisions with C++ using the ROOT framework
	\end{itemize}
	 



\section{Posters and Presentations}

{\bf APS Division of Nuclear Physics} \hfill Oct.  2015\\
Monte Carlo Investigations of Quark Gluon Plasma 
\begin{itemize} \itemsep -2pt  % reduce space between items
	\item Presented research from Summer 2015 as a poster 
\end{itemize}

{\bf Celebration of Undergraduate Research Oberlin College} \hfill Sept. 2015 \\
Monte Carlo Investigations of Quark Gluon Plasma\\
Presented as a poster and 15 minute talk

		\section{Graduate Coursework}

{\bf Statistical Mechanics} - Physics\hfill Fall  2017\\ 
{\bf Computational Physics} - Applied Computation\hfill Fall  2017\\ 
{\bf Stochastic Processes and Disordered Systems} - Applied Math\hfill Spring  2018\\ 
{\bf Inverse Problems In Science and Engineering} - Applied Math\hfill Spring  2018\\ 
{\bf Advanced Machine Learning} - Computer Science\hfill Fall  2018\\ 
{\bf Advanced Quantum Mechanics} - Physics\hfill Fall  2018\\ 
{\bf Evolutionary Dynamics} - Math\hfill Spring 2019\\ 





	
	\section{Work and Teaching Experience}
	{\bf Teaching Fellow} Applied Math 50, Harvard University \hfill Spring 2019
	\begin{itemize}
		\item Helped develop a significant portion of the homeworks and in class labs
		\item I hold section and office hours weekly to convey ideas in programming and applied math
	\end{itemize}
	{\bf ExCo Instructor} 3D Printing \& Design \hfill\small {Fall 2015[6,7] Spring 2016/2017}
	\normalsize\\
	At Obelin the ExCo program is a student-run, for-credit Experimental College with courses taught by students or community members.
	\begin{itemize} \itemsep -2pt  % reduce space between items
		\item Developed a Course centered on 3D printing technologies
		\item Themes included building, usingl and maintaining 3D printers\\ as well as basic electrical engineering
	\end{itemize}
	 
	{\bf Science Outreach} Oberlin Boys and Girls Club \hfill Sept. 2014 - May 2015
	\begin{itemize} \itemsep -2pt  % reduce space between items
		\item Co-Founded Science program for Boys and Girls Club
		\item Developed and ran interactive science demos for 3rd-5th grade students
	\end{itemize}
	
	{\bf Tutoring} Oberlin College \hfill Fall 2014-Spring 2017 \\
	\vspace{-4mm}
	\begin{itemize} \itemsep -2pt
		\item Personal Tutor for High School IB math student
		\item Tutored Oberlin College Students in
		\begin{itemize} \itemsep -2pt  % reduce space between items
			\item Introductory Calculus
			\item Multivariate Calculus
			\item Introductory Economics
			\item Modern Physics
			\item Drop In Calculus
		\end{itemize}
	\end{itemize}
	\vspace{-2mm}

	
	 
	{\bf Counselor} Riverbend Environment Education Center \hfill Summer 2014
	\begin{itemize} \itemsep -2pt  % reduce space between items
		\item Led groups of children in  a team with a co-counselor
		\item Designed and implemented lesson plans
	\end{itemize}
	 
	
	

	 \section{Distinctions}
	 \begin{itemize} \itemsep -2pt
		\item Member Sigma Xi
		\item Member Phi Beta Kappa
		\item One of four Oberlin nominees for Goldwater Scholarship in 2016
		\item John F. Oberlin Scholarship recipient
		%\item 3\textsuperscript{rd} Degree Black Belt from AmKor Karate. (Training since 2004)
	\end{itemize}
	\section{Leadership   Activities} 
	{\bf Treasurer and Director} OC3D Oberlin College 3D Printing    \hfill         Sept. 2015- May 2017 
	\begin{itemize} \itemsep -2pt
		\item  Manage Club Accounts
		\item  Increase Club Membership
		\item  Organize Club Projects
		\item Involved in developing the OC3D space since Jan. 2014
	\end{itemize}
	
	\section{Online Presence}
	\begin{itemize} \itemsep -2pt  % reduce space between items
		\item Stackoverflow: stackoverflow.com/users/835607/ianhi
		\item Github:  github.com/ianhi 
	\end{itemize}


	
\end{resume} 
\end{document} 

